% In this file you should put the actual content of the blueprint.
% It will be used both by the web and the print version.
% It should *not* include the \begin{document}
%
% If you want to split the blueprint content into several files then
% the current file can be a simple sequence of \input. Otherwise It
% can start with a \section or \chapter for instance.

\chapter{Faithfully Flat Modules}

\begin{definition}\cite[\href{https://stacks.math.columbia.edu/tag/00HB}{Definition 00HB}]{stacks-project}
	\label{definition-flat}
	Let $R$ be a ring.
	\begin{enumerate}
		
		\item An $R$-module $M$ is called {\it faithfully flat} if the
		complex of $R$-modules
		$N_1 \to N_2 \to N_3$ is exact if and only if
		the sequence $M \otimes_R N_1 \to M \otimes_R N_2 \to M \otimes_R N_3$
		is exact.
		
		\item A ring map $R \to S$ is called {\it faithfully flat} if
		$S$ is faithfully flat as an $R$-module.
	\end{enumerate}
\end{definition}

\begin{lemma}
	\label{lemma-composition-flat}
	A composition of (faithfully) flat ring maps is
	(faithfully) flat.
	If $R \to R'$ is (faithfully) flat, and $M'$ is a
	(faithfully) flat $R'$-module, then $M'$ is a
	(faithfully) flat $R$-module.
\end{lemma}

\begin{proof}
	The first statement of the lemma is a particular case of the
	second, so it is clearly enough to prove the latter. Let
	$R \to R'$ be a flat ring map, and $M'$ a flat $R'$-module.
	We need to prove that $M'$ is a flat $R$-module. Let
	$N_1 \to N_2 \to N_3$ be an exact complex of $R$-modules.
	Then, the complex $R' \otimes_R N_1 \to
	R' \otimes_R N_2 \to R' \otimes_R N_3$ is exact (since $R'$
	is flat as an $R$-module), and so the complex
	$M' \otimes_{R'} \left(R' \otimes_R N_1\right)
	\to M' \otimes_{R'} \left(R' \otimes_R N_2\right)
	\to M' \otimes_{R'} \left(R' \otimes_R N_3\right)$ is
	exact (since $M'$ is a flat $R'$-module). Since
	$M' \otimes_{R'} \left(R' \otimes_R N\right)
	\cong \left(M' \otimes_{R'} R'\right) \otimes_R N
	\cong M' \otimes_R N$ for any $R$-module $N$ functorially
	(by Lemmas 12.7 and
	12.3), this complex is isomorphic
	to the complex
	$M' \otimes_R N_1 \to M' \otimes_R N_2 \to M' \otimes_R N_3$,
	which is therefore also exact. This shows that $M'$ is a flat
	$R$-module. Tracing this argument backwards, we can show
	that if $R \to R'$ is faithfully flat, and if $M'$ is
	faithfully flat as an $R'$-module, then $M'$ is faithfully
	flat as an $R$-module.
\end{proof}

\begin{lemma}
	\label{lemma-flat-base-change}
	Suppose that $M$ is faithfully flat over $R$, and that $R \to R'$
	is a ring map. Then $M \otimes_R R'$ is faithfully flat over $R'$.
\end{lemma}

\begin{proof}
	For any $R'$-module $N$ we have a canonical
	isomorphism $N \otimes_{R'} (R'\otimes_R M)
	= N \otimes_R M$. Hence the desired exactness properties of the functor
	$-\otimes_{R'}(R'\otimes_R M)$ follow from
	the corresponding exactness properties of the functor $-\otimes_R M$.
\end{proof}

\begin{lemma}
	\label{lemma-flatness-descends}
	Let $R \to R'$ be a faithfully flat ring map.
	Let $M$ be a module over $R$, and set $M' = R' \otimes_R M$.
	Then $M$ is flat over $R$ if and only if $M'$ is flat over $R'$.
\end{lemma}

\begin{proof}
	By Lemma \ref{lemma-flat-base-change} we see that if $M$ is flat
	then $M'$ is flat. For the converse, suppose that $M'$ is flat.
	Let $N_1 \to N_2 \to N_3$ be an exact sequence of $R$-modules.
	We want to show that $N_1 \otimes_R M \to N_2 \otimes_R M \to N_3 \otimes_R M$
	is exact. We know that
	$N_1 \otimes_R R' \to N_2 \otimes_R R' \to N_3 \otimes_R R'$ is
	exact, because $R \to R'$ is flat. Flatness of $M'$ implies that
	$N_1 \otimes_R R' \otimes_{R'} M'
	\to N_2 \otimes_R R' \otimes_{R'} M'
	\to N_3 \otimes_R R' \otimes_{R'} M'$ is exact.
	We may write this as
	$N_1 \otimes_R M \otimes_R R'
	\to N_2 \otimes_R M \otimes_R R'
	\to N_3 \otimes_R M \otimes_R R'$.
	Finally, faithful flatness implies that
	$N_1 \otimes_R M \to N_2 \otimes_R M \to N_3 \otimes_R M$
	is exact.
\end{proof}

\begin{lemma}
	\label{lemma-flatness-descends-more-general}
	Let $R$ be a ring. Let $S \to S'$ be a flat map of $R$-algebras.
	Let $M$ be a module over $S$, and set $M' = S' \otimes_S M$.
	\begin{enumerate}
		\item If $M$ is flat over $R$, then $M'$ is flat over $R$.
		\item If $S \to S'$ is faithfully flat, then $M$ is flat
		over $R$ if and only if $M'$ is flat over $R$.
	\end{enumerate}
\end{lemma}

\begin{proof}
	Let $N \to N'$ be an injection of $R$-modules. By the flatness
	of $S \to S'$ we have
	$$
	\Ker(N \otimes_R M \to N' \otimes_R M) \otimes_S S'
	=
	\Ker(N \otimes_R M' \to N' \otimes_R M')
	$$
	If $M$ is flat over $R$, then the left hand side is zero and
	we find that $M'$ is flat over $R$ by the second characterization
	of flatness in Lemma 39.5.
	If $M'$ is flat over $R$ then we have the vanishing of the right hand side
	and if in addition $S \to S'$ is faithfully flat, this implies that
	$\Ker(N \otimes_R M \to N' \otimes_R M)$ is zero which in turn
	shows that $M$ is flat over $R$.
\end{proof}

\begin{lemma}
	\label{lemma-flat-permanence}
	Let $R \to S$ be a ring map. Let $M$ be an $S$-module.
	If $M$ is flat as an $R$-module and faithfully flat as an $S$-module,
	then $R \to S$ is flat.
\end{lemma}

\begin{proof}
	Let $N_1 \to N_2 \to N_3$ be an exact sequence of $R$-modules.
	By assumption $N_1 \otimes_R M \to N_2 \otimes_R M \to N_3 \otimes_R M$
	is exact. We may write this as
	$$
	N_1 \otimes_R S \otimes_S M
	\to
	N_2 \otimes_R S \otimes_S M
	\to
	N_3 \otimes_R S \otimes_S M.
	$$
	By faithful flatness of $M$ over $S$ we conclude that
	$N_1 \otimes_R S \to N_2 \otimes_R S \to N_3 \otimes_R S$ is exact.
	Hence $R \to S$ is flat.
\end{proof}

\begin{lemma}
	\label{lemma-easy-ff}
	Let $R$ be a ring.
	Let $M$ be an $R$-module.
	The following are equivalent
	\begin{enumerate}
		\item $M$ is faithfully flat, and
		\item $M$ is flat and for all $R$-module homomorphisms $\alpha : N \to N'$
		we have $\alpha = 0$ if and only if $\alpha \otimes \text{id}_M = 0$.
	\end{enumerate}
\end{lemma}

\begin{proof}
	If $M$ is faithfully flat, then
	$0 \to \Ker(\alpha) \to N \to N'$ is exact if and only if the same holds
	after tensoring with $M$. This proves (1) implies (2).
	For the other, assume (2). Let $N_1 \to N_2 \to N_3$
	be a complex, and assume the complex
	$N_1 \otimes_R M \to N_2 \otimes_R M \to N_3\otimes_R M$
	is exact. Take $x \in \Ker(N_2 \to N_3)$,
	and consider the map $\alpha : R \to N_2/\Im(N_1)$,
	$r \mapsto rx + \Im(N_1)$. By the exactness
	of the complex $-\otimes_R M$ we see that $\alpha \otimes
	\text{id}_M$ is zero. By assumption we get that $\alpha$ is
	zero. Hence $x $ is in the image of $N_1 \to N_2$.
\end{proof}

\begin{lemma}
	\label{lemma-ff}
	\begin{slogan}
		A flat module is faithfully flat if and only if it has nonzero fibers.
	\end{slogan}
	Let $M$ be a flat $R$-module.
	The following are equivalent:
	\begin{enumerate}
		\item $M$ is faithfully flat,
		\item for every nonzero $R$-module $N$, then tensor product $M \otimes_R N$
		is nonzero,
		\item for all $\mathfrak p \in \Spec(R)$
		the tensor product $M \otimes_R \kappa(\mathfrak p)$ is nonzero, and
		\item for all maximal ideals $\mathfrak m$ of $R$
		the tensor product $M \otimes_R \kappa(\mathfrak m) = M/{\mathfrak m}M$
		is nonzero.
	\end{enumerate}
\end{lemma}

\begin{proof}
	Assume $M$ faithfully flat and $N \not = 0$. By Lemma \ref{lemma-easy-ff}
	the nonzero map $1 : N \to N$ induces a nonzero map
	$M \otimes_R N \to M \otimes_R N$, so $M \otimes_R N \not = 0$.
	Thus (1) implies (2). The implications (2) $\Rightarrow$ (3) $\Rightarrow$ (4)
	are immediate.
	
	\medskip\noindent
	Assume (4). Suppose that $N_1 \to N_2 \to N_3$ is a complex and
	suppose that $N_1 \otimes_R M \to N_2\otimes_R M \to
	N_3\otimes_R M$ is exact. Let $H$ be the cohomology of the complex,
	so $H = \Ker(N_2 \to N_3)/\Im(N_1 \to N_2)$. To finish the proof
	we will show $H = 0$. By flatness we see that $H \otimes_R M = 0$.
	Take $x \in H$ and let $I = \{f \in R \mid fx = 0 \}$
	be its annihilator. Since $R/I \subset H$ we get
	$M/IM \subset H \otimes_R M = 0$ by flatness of $M$.
	If $I \not =  R$ we may choose
	a maximal ideal $I \subset \mathfrak m \subset R$.
	This immediately gives a contradiction.
\end{proof}

\bibliographystyle{plain}
\bibliography{ref}